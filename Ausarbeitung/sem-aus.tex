\begin{filecontents*}{\jobname.bib}
@article{Worsch_2009_AUC_ar,
  author  = {Thomas Worsch and Hidenosuke Nishio},
  title   = {Achieving universality of {CA} by changing the neighborhood},
  journal = {Journal of Cellular Automata},
  year    = {2009},
  volume  = {4},
  number  = {3},
  pages   = {237--246},
}
@inproceedings{Worsch_2012_IUA_ip_acri,
  author    = {Thomas Worsch},
  title     = {({I}ntrinsically?) Universal Asynchronous Cellular Automata},
  editor    = {Georgios Sirakoulis and Stefania Bandini},
  booktitle = {Proceedings ACRI 2012},
  year      = {2012},
  pages     = {689--698},
  publisher = {Springer},
  series    = {LNCS},
  volume    = {7495},
}
\end{filecontents*}

\documentclass[11pt,a4paper]{article}

\usepackage[T1]{fontenc}
\usepackage[ngerman]{babel}
\usepackage[utf8]{inputenc}

\usepackage{amsmath}
% \usepackage{mathtools}

\usepackage[tt=false]{libertine}   % !!!!! das muss man nicht nutzen
\usepackage[libertine]{newtxmath}  % !!!!! das muss man nicht nutzen
%\usepackage[supstfm=libertinesups,supscaled=1.2,raised=-.13em]{superiors} % params taken from doc

%
%\usepackage{tgpagella}
%\usepackage[euler-digits]{eulervm}
%
\usepackage{csquotes}

\usepackage{microtype}

\usepackage{fancyvrb}

\usepackage{graphicx}

\usepackage{booktabs}
\usepackage[shortlabels]{enumitem}
\setlist{noitemsep}

\usepackage{titlesec}
\usepackage{tcolorbox}
\tcbuselibrary{listingsutf8}

\usepackage{bbold}
\newcommand{\Z}{\mathbb{Z}}

\usepackage{hologo}

%-----------------------------------------------------------------------------
% für das Deckblatt

\usepackage{tikz}

\newcommand{\teilnehmername}{Klaus Philipp Theyssen} % !!!!!
\newcommand{\teilnehmermatrnr}{2061578}        % !!!!!
\newcommand{\seminarart}{Proseminar}           % !!!!!  oder Seminar
\newcommand{\seminarlp}{3 LP}                  % !!!!!  Prosem: immer 3 LP, Sem: 3 oder 4 LP
\newcommand{\seminarjahr}{2019}                % !!!!!
%-----------------------------------------------------------------------------
\newcommand{\meta}[1]{$\langle$\textit{#1}$\rangle$}
\newcommand{\paket}[1]{\texttt{#1}}
\newcommand{\prgname}[1]{\texttt{#1}}

%-----------------------------------------------------------------------------
\author{Klaus Philipp Theyssen}
\title{Proseminar Ausarbeitung Brown'sche Schaltkreise}

%=============================================================================
\begin{document}
%=======================================================================
% Anfang erste Seite
{\thispagestyle{empty}\large\sffamily\raggedright
%
\begin{tikzpicture}[remember picture,overlay]
  \coordinate[xshift=5mm,yshift=-5mm] (NW) at (current page.north west) {};
  \coordinate[xshift=-5mm,yshift=-5mm] (NE) at (current page.north east) {};
  \coordinate[xshift=-5mm,yshift=13mm] (SE) at (current page.south east) {};
  \coordinate[xshift=5mm,yshift=13mm] (SW) at (current page.south west) {};

  \draw[line width=0.25pt] (NW)
    [rounded corners=5mm] -- (NE) 
    [sharp corners] -- (SE)
    [rounded corners=5mm] -- (SW)
    [sharp corners] -- cycle
  ;
\end{tikzpicture}
%
\unskip % keine Ahnung warum das nötig ist
\noindent \textbf{\Large \seminarart\ (\seminarlp)} 
\\[\baselineskip]
%
Zellularautomaten und diskrete komplexe Systeme
% für Fortgeschrittene  % nur für das 4 Leistungspunkte Seminar !!!!!
\\[1ex]
%
im Sommersemester \seminarjahr

\vspace*{3\baselineskip}

\noindent \textbf{\Large Ausarbeitung} \\[\baselineskip]
%
von \textbf{\teilnehmername}, Matr.nr.~\teilnehmermatrnr

\vspace*{3\baselineskip}

\noindent \textbf{\Large Thema} \\[\baselineskip]
%
% nachfolgende ein Beispiel, für Konferenzbeiträge, Buchausschnitte, ...
% bitte analog vorgehen !!!!!
%
 Ferdinand Peper and Jia Lee (2018)\\[1ex]
%
\textit{On Non-polar Token-Pass Brownian Circuits}\\[1ex]
%
Reversibility and Universality, S.299-311
}
\clearpage
% Ende erste Seite
%=======================================================================
% Anfang zweite Seite
{\thispagestyle{empty}\raggedright

\noindent \textbf{\Large Erklärung}\\[1ex]
gemäß \S 6 (11) der Prüfungsordnung Informatik % !!!!! oder \S 6 (7) (bei MasterPO 2015)
(Bachelor) 2015 % oder Master !!!!!
\\[\baselineskip]

\noindent
Ich versichere wahrheitsgemäß, die Seminarausarbeitung zum
\seminarart{} "`Zellularautomaten und diskrete komplexe Systeme"' im
Sommersemester \seminarjahr{} selbstständig angefertigt, alle
benutzten Hilfsmittel vollständig und genau angegeben und alles
kenntlich gemacht zu haben, was aus Arbeiten anderer unverändert oder
mit Abänderungen entnommen wurde.

\vspace*{30mm}
\noindent
\begin{tabular}{@{}l}
  \hline
   \\[-1ex]
  \hbox to 0.6\textwidth{(\teilnehmername, Matr.nr.~\teilnehmermatrnr) \hss}
\end{tabular}
}
\clearpage
% Ende zweite Seite
%=======================================================================

%-----------------------------------------------------------------------------
\section{Einführung}
Bei Elektronik im nanometer Bereich sind Rauschen und Fluktuation 
entscheidende Faktoren die beim Entwurf entsprechender Schaltkreise 
zu beachten sind.
%
Desweitern ist ein geringerer Energieverbrauch anzustreben, daher 
werden in Zukunft Geräte von Interesse sein die nur von einzelnen 
Partikeln geschaltet werden.
%
Die in diesem Paper präsentiereten Brown'schen Schaltkreise nutzen Tokens
als Signale und setzen Fluktuation aktiv bei ihren Berechnungen ein.
Die Fluktuation orientiert sich dabei an der Brown'schen Molekularbewegung 
in der Biologie.
%
Im Paper Fundamentals: Brownian Circuits[2] wurden Brown'sche Schaltkreise 
vorgestellt und das T-Element, welches für die Klasse von Schaltkreisen 
universell ist, eingeführt.



%-------------------------------------------------------------------------------
\section{Grundlagen}
Tokens sind diskrete nicht teilbare Einheiten die Signale modellieren.
%
Zunächst werden die im Paper behandelten Schaltkreis Typen vorgestellt 
um Unterschiede in Funktionalität und Aufbau hervorzuheben. 
%
Dann wird das T-Element betrachtet und wie sich damit die Universaltät der
Brown'schen Token pass Schaltkreise ergibt.
%
Allgemein sind alle hier vorgestellten Schaltkreise asynchron, dies bedeutet sie
haben keinen Zeitgeber und es kann nebenläufig zu Änderungen am Signal kommen.
%
Sie sind robust gegen Verzögerungen (delay-insentive), was heißt das Verzögerungen in der Signalweiterleitung
nicht zu unkorrekten Berechnungen führen.


\subsection{Token-based Schaltkreise}
Token basierte Schaltkreise sind wichtig für physikalische Implementierungen bei
denen Signale diskret sind.
%
Dabei basieren die Schaltungen auf der Bewegung
von einzelnen Ladungen. 
%
Ein Beispiel für token basierte Schaltkreise sind Petrinetze. 
%
Token basierte Schaltkreise die delay insentive sind können mithilfe 
einer Menge von Schaltkreisprimitiven konstruiert werden, genauso 
wie synchrone Schaltkreise aus NOT-Gattern und UND-Gattern.
%
Eine solche Menge nennt man dann universel und Abbildung 1 gibt 
dafür ein Beispiel.
%
Dabei führt Merge zwei Tokens zu einem zusammen.
% 
Fork macht aus einem Token zwei.
%
Tria fügt zwei Tokens zusammen und je nach eingabe Kabel kommt
die Ausgabe auf ein bestimmtes Kabel.
%
Wobei das Zusammenführen beim Tria nur funktioniert wenn zwei Tokens da sind,
also ein einzelnes Token wartet solange bis an zweites kommt bis die 
Überführung stattfindet.
%
Diese Funktionalität in asynchronen Schaltkreisen entspricht dem Takt in
synchronen Schaltkreises und ist deshalb hervorzuheben.
%

\begin{figure}[h]
       \centering
       \includegraphics[width=9cm]{bilder/tokenBased.png}
       \caption{Merge, Fork und Tria}
\end{figure}    

%

\subsection{Brown'sche Schaltkreise}
Sie haben wie token-basierte Schaltkreise auch Tokens die sich auf Kabeln bewegen
und in Schaltkreiselementen miteinander interagieren.
%
Allerdings wird hier die Interaktion durch Fluktuation getrieben und dient
als treibende Kraft für das Zusammenwirken der Tokens innerhalb der
Schaltkreiselemente. 
%
Dies ermöglicht Deadlocks mithilfe 
von Backtracking aufzulösen, was sich in einfacherem Design wiederspiegelt.


\subsection{Token-pass Schaltkreise}
Der Name kommt von der Bauweise dieser Schaltkreise, sie verbinden einfach nur Kabel 
miteinander durch die Tokens hindurchlaufen.
%
Token-pass Schaltkreisen lassen die Zahl der Tokens gleich.
%
Tokens können nicht entstehen oder verschwinden und auch 
nicht auf andere Kabel wechseln.
%
Äquivalenz von Token-pass und token-based zeigen (ein kabel wird zu zwei)
die entsprechenden TP-Merge, TP-Fork, TP-Tria.
%
Token-pass Schaltkreise haben eine Menge an EingabeKablen die in
den Schaltkreis führen und eine Menge an Ausgabekabeln.
%
Dabei können innerhalb des Schaltkreises
noch Schleifen sein. 
%

\begin{figure}[h]
    \centering
    \includegraphics[width=8cm]{bilder/TokenPassScheme.png}
    \caption{Token Pass Schema}
\end{figure} 

\subsubsection{Polare token-pass Schaltkreise}
In polaren token-pass Schaltkreises existiert eine bevorzugte Richtung
der Token, gekennzeichnet durch einen Pfeil.
%
Besonders bei den pre-Kabeln und post-Kabeln (also für Ein- und Ausgabe)
ist dies sinnvoll.

\subsubsection{Nicht polare token-pass Schaltkreise}
Hier können die Tokens frei fluktuieren, allerdings haben die T-Elemente eine
Einschränkung (kreise und blank symbole) wie sie Tokens verarbeiten.
%
Außerdem gibt es hier die möglichkeit von Terminatoren, dies sind Kabel
mit einem Ende auf dem Tokens sich einfach nur vor und zurück bewegen.
%
Die nicht-polaren Schaltkreise ermöglichen einfacheres Design und Verwendung
von weniger T-Elementen, weil bestimmtes Verhalten zum Verhindern von
Deadlocks nicht expizit modelliert werden muss.

%TODO T-Elemente bei Grundlagen mit einbauen und bei den jeweiligen 
%     Schaltkreis arten darauf verweisen
%-----------------------------------------------------------------------------
\section{T-Element}
Grundlegende Funktion des T-Elementes entspricht mit Abbildung 3.
%
Eingang c ist der Basis Eingang des T-Elmentes hier muss immer ein Token
anliegen damit es zur Verarbeitung kommt.
%
Wenn jetzt bei c ein Token anliegt und bei einem der beiden anderen
Eingänge a oder b noch ein Token anliegt werden diese vom T-Element 
entlang des gestrichelten Halbkreises auf das parallel verlaufende 
Kabel überführt.
%
Wenn bei a und b ein Token anliegt wird zufällig 
eines der beiden ausgewählt und mit c überführt.

\begin{figure}[h]
    \centering
    \includegraphics[width=7cm]{bilder/T_Element.png}
    \caption{T-Element}
\end{figure}    

In Abbildung 4 ist zu erkennen wie die token basierten Schaltkreispirmitive 
(Merge, Fork und Tria) mit 

\begin{figure}[h]
    \centering
    \includegraphics[width=7cm]{bilder/BasedToPass.png}
    \caption{Äquivalenz Token based Token pass}
\end{figure}    

\subsection{Universalität}

%-----------------------------------------------------------------------------
\section{1-Bit Speicher}
Nun soll anhand eines 1-Bit Speichers die Funktionsweise von brown'schen 
token-pass Schaltkreisen erläutert werden. 

\begin{figure}[h]
      \centering
      \includegraphics[width=9cm]{bilder/NonPolarMemory.png} 
      \caption{1-Bit Speicher nicht polar token pass}
\end{figure}

\subsection{Polarer token-pass 1-Bit Speicher}
Es werden 8 T-Elemente benötigt, es gibt ein Write und ein Read Bereich
es kann zu
%TODO was ist gemeint 
Überlaufen??!!
kommen beim schreiben.  

\subsection{Nicht polarer token-pass 1-Bit Speicher}
Es werden nur 7 T-Elemente benötigt auch, hier Konzept von Lesen und Schreiben
erklären und Bedeutung/ Nutzen von Terminator Kabeln.
Das Deadlock Backtracking zeigen.

%-----------------------------------------------------------------------------
\section{UND-Bauteil}

\begin{figure}[h]
    \centering
    \includegraphics[width=12cm]{bilder/UndGatter.png}
    \caption{UND-Gatter aus 11 T-Elementen}
\end{figure}    


Als Teil meiner Eigenarbeit im Rahmen dieses Proseminars habe ich ein UND-Gatter
mithilfe von nicht-polaren T-Elementen entworfen.
%
Es benutzt den Deadlock backtracking Mechanismus und verundet ansonsten jede 
mögliche Eingabe mit jeder möglichen Ausgaben. 
%
Dabei werden zunächst T-Element zum modellieren der möglichen Wege benutzt, also 
das Token kann einen davon wählen, was wiederum der richtige ist oder zu einem Deadlock
führt. Solange bis sich beide Input Tokens "gefunden" haben.
%
Dann werden bei der C0 Ausgabe zwei Tokens benutzt um einfach die möglichen Wege zu 
verbinden.


\subsection{Initialisierung}
Es ist eine Initialisierung auf der Abbildung gegeben (die Position der Tokens), 
außerdem ist sind die Kreise in den T-Elementen für eine korrekte Berechnung
elementar.
%
Diese Initialisierung ist natürlich nicht eindeutig und auch die Anordnung der Elemente 
ist veränderbar was die Funktion nicht beeinflusst.

\subsection{Korrektheit}
Eine interessante Frage ist nun ob die Korrektheit dieses Schaltkreises beweisbar ist. 
Wenn wir die angegebene Initialisierung vorraussetzen können, können wir uns die 
Korrektheit schnell klar machen indem :
%TODO mögliche Token interaktionen druchspielen OBdA für einen Fall



% ----------------------------------------------------------------------------
\section{Zusammenfassung und Ausblick}
In dem Paper (Non-polar token-pass Brownian Circuits) [1] wird eine neue Art
von Schaltkreis vorgestellt der zukünfitig in der Nanoelektronik eingesetzt
werden könnte.
%
Dabei ist das Konzept von Brown'scher Bewegung der Signale (Tokens) 
der interessante und neue Aspekt der es ermöglicht neue Arten von Schaltkreisen 
zu designen und auf ihre Eigenschaften zu untersuchen.
%
Jedoch sind Dinge wie Geschwindigkeit der Berechnung, Korrektheit beweisen und 
welche arten von konkreten Implementierungen möglich sind noch zu klären.


\subsection{Wann ist Berechnung vorbei?}

\subsection{Geschwindigkeit der Berechnung}

\subsubsection{Ein langes Kabel vs. viele T-Elemente}

\subsection{Implementierung}




\label{sec:summary}

Zum Abschluss kommt das Literaturverzeichnis.
%
Die beiden Zeilen

\begin{tcblisting}{listing only}
  \bibliographystyle{plain}
  \bibliography{\jobname}
\end{tcblisting}

erzeugen das, was man unter dieser Zeile sieht:

\bibliographystyle{plain}
\bibliography{\jobname}

\end{document}

%% Encoding: ISO8859-1 %%

\section{Grundlagen}
\frame{
\frametitle{Grundlagen}
    \begin{itemize}
        \item token basierte Schaltkreise (Petri Netze)
        \item token pass Schaltkreise
        \item nicht polare token pass Brown'sche Schaltkreise
    \end{itemize}
}

 %TODO figure for token based
\frame{
    \frametitle{Token basierte Schaltkreise}
    \begin{itemize}
        \item Petri Netze sind ein Beispiel
        \item Signal als Token 
    \end{itemize}
   
}

%TODO insert figure of token pass scheme
\frame{
    \frametitle{Token pass Schaltkreise}
    \begin{itemize}
        \item Einschrankungen von token basiert
        \item Anzahl der Tokens immer gleich
    \end{itemize}
   
}    

\frame{
    \frametitle{Aquivalenz von token basiert und brown'schen token pass}
   %TODO insert figure of äquivalenz
}

\section{1-Bit Memory}
%TODO Funktionsweise auf jeden Fall von nicht polarem, aber 
% soll ich auch das polare erklären ???
%für bilder die abläufe aus dem Paper benutzen



\section{UND-Gatter} 





\section{Ausblick} 




